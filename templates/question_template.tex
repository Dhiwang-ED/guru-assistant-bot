\documentclass[a4paper]{article}
\usepackage[utf8]{inputenc}
\usepackage{graphicx} % Paket untuk menyertakan gambar
\usepackage[left=2cm,right=2cm,top=2.5cm,bottom=2cm]{geometry} % Mengatur margin halaman
\usepackage{fancyhdr} % Paket untuk header/footer kustom (opsional, untuk tampilan lebih canggih)
\pagestyle{plain} % Nonaktifkan header/footer bawaan jika pakai fancyhdr

\begin{document}

% --- Header Halaman ---
\begin{center}
    % Pastikan path relatif ke gambar Anda benar dari lokasi kompilasi LaTeX!
    % Jika bot_scripts/generate_question.py menjalankan pdflatex di /tmp,
    % dan folder assets/images ada di root repo, maka pathnya adalah:
    % /workspaces/guru-assistant-bot/assets/images/logo_sekolah.png
    % Untuk pdflatex di /tmp, kita bisa menggunakan path absolut atau symlink
    % Alternatif yang lebih robust: menyalin gambar ke /tmp sebelum kompilasi.
    % Namun, untuk POC, kita asumsikan path relatif ini bisa bekerja jika pdflatex dijalankan dari root repo
    % Atau paling aman, salin gambar ke /tmp lalu panggil langsung namanya.
    \includegraphics[width=0.2\textwidth]{../assets/images/logo_sekolah.png} \\ % PATH INI AKAN KITA UBAH DI PYTHON SCRIPT
    \vspace{0.5cm} % Spasi vertikal
    {\large Mata Pelajaran: \textbf{ {{ subject }} }} \\
    {\large Kelas: \textbf{ {{ grade }} }} \\
    {\large Guru: \textbf{ {{ teacher_name }} }} \\
    \vspace{0.2cm}
    \rule{\textwidth}{1pt} % Garis pemisah
    \vspace{0.2cm}
    {\Large \textbf{ {{ exam_title }} }} \\
    {\small Tanggal: {{ date }} }
\end{center}

\vspace{1cm}

% --- Isi Soal (dari Jinja2) ---
\textbf{PETUNJUK UMUM:}
\begin{enumerate}
    \item Bacalah setiap soal dengan teliti.
    \item Jawablah semua pertanyaan pada lembar jawaban yang tersedia.
\end{enumerate}

\section*{Bagian A: Pilihan Ganda}

\begin{enumerate}
    \item Soal pilihan ganda pertama:
        \begin{enumerate}[A.]
            \item Pilihan A
            \item Pilihan B
            \item Pilihan C
            \item Pilihan D
            \item Pilihan E
        \end{enumerate}
    \item Soal pilihan ganda kedua:
        \begin{enumerate}[A.]
            \item Pilihan A
            \item Pilihan B
            \item Pilihan C
            \item Pilihan D
            \item Pilihan E
        \end{enumerate}
\end{enumerate}

\section*{Bagian B: Esai}

\begin{enumerate}
    \item Jelaskan konsep {{ topic_1 }} secara singkat!
    \item Bagaimana hubungan antara {{ topic_2 }} dan {{ topic_3 }}?
\end{enumerate}

\end{document}